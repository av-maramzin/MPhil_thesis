\chapter{Introduction}
\label{introduction}

\section{Overview}
\label{introduction_overview}
% The importance of software parallelization
\quad Parallelism has become pervasive in the world of computing, with parallel hardware omnipresent across the whole spectrum of various computing systems from low-end embedded processors to high-end supercomputers. Yet, most of the existing software is written in a sequential fashion: be it an old legacy software initially designed to run on the available at that time serial hardware or modern applications being developed by application domain experts rather than performance engineers. In order to exploit all available hardware facilities software has to be parallelized.
% Challenges in the field
\begin{description}[style=unboxed,leftmargin=0cm]
\item[Manual parallelization] The task of software parallelization has characteristically been a very manual process, which is multifaceted, extremely complex, time consuming and error-prone. An embarrassingly parallel problem might end up being hidden behind a thoughtless software design, a serial algorithm or being implemented with an unsuccessfully chosen lover level constructs, such as pointers, heap-allocated and pointer-linked data structures, indirect array referencing, etc. To elegantly and effectively map a parallel problem onto the exact hardware a programmer must work on the various abstract levels and possess an expert-level knowledge in the range of various fields from software design and algorithmic patterns in software engineering down to compiler's automatic vectorization and hardware cache coherence protocols. It is not always realistic to expect such a deep expertise from an average programmer. Done in the wrong way software parallelization can even slow the program down in comparison to its original sequential version. 
\item[Automatic parallelization] Despite decades of intensive research in automatic software parallelization~\cite{6813266}, fully exploiting the potential of modern multi- and many-core hardware still requires a significant manual effort. Furthermore, the automatic parallelization techniques are limited to narrow domains of scientific Fortran codes and relatively simple computational idioms.
\item[ML based parallelization] There has been a wide research into utilizing more exotic to the field of software parallelization machine learning based methods. A good overview is provided by \cite{ml-oboyle}. Although these methods have proved to be extremely useful and high performing on some compilation technology problems like selecting the best compiler flags or finding the most optimal compiler optimization parameters (like loop unroll or function inline factors). Due to the inherent statistical errors and unavailability of large training data sets these methods have not yet found a widespread application in the area of software parallelization. 
\end{description}
% The solution
We have done a thorough literature review and understand the background. In our project we are not trying to find a "silver bullet" and solve the problem of automatic parallelization. Neither do we try to tune a machine learning algorithms to a 100\% perfect prediction accuracy. Given the difficulty of the obstacles faced by the field today, we do not expect that programmers will be liberated from performing manual parallelization in the near future~\cite{Larsen:2012:PML:2410141.2410600}. Instead, we acknowledge the role of a human programmer in the software parallelization process, but we do not expect the programmer to be an expert. All we try to do is to reduce the manual effort by providing a programmer with a parallelization \textit{assistant solution}. Our solution alleviates the task and makes it more accessible for an average programmer. The assistant solution we propose is as multifaceted as the problem itself. To fully exploit all the potential of software parallelization a programmer has to work on several conceptual levels. Thus, the assistant solution consists of a tool and a library aiming at different stages of software parallelization process. 

\section{Loop parallelization assistant}
\label{introduction_assistant}

% problem: the need for manual parallelization, auto-parallelisation does not deliver desired performance improvements

\quad Despite decades of intensive research in automatic software
parallelization~\cite{6813266}, fully exploiting the potential of modern multi- and many-core hardware still requires a significant manual effort.

% solution: parallelisation assistant

Chapter \ref{assistant} introduces a novel parallelization assistant tool that aids a programmer in the process of parallelizing a program in the frequent case where automatic approaches fail to do so. The assistant works at the finer levels of granularity, namely the program loops. Loops are compelling candidates for parallelization, as they are naturally decomposable and tend to capture most of the execution time in a program.
%
The assistant reduces the manual effort in this process by presenting a programmer with a ranking of program loops that are most likely to 1) require little or no effort for successful parallelization and 2) improve the program's performance when parallelized.
%
Thus, it improves over the traditional, profile-guided process by also taking into account the \emph{probability} of potential parallelization for each of the profiled loops.

% how?

At the core of our parallelization assistant resides a novel machine-learning (ML) model of loop parallelizability.
%
Focusing on loops allows the model to leverage a large amount of specific analyses available in modern compilers, such as generalized iterator recognition~\cite{Manilov:2018:GPI:3178372.3179511} and loop dependence analysis~\cite{Jensen:2017:ILD:3132652.3095754}.
%
The model encodes the results of these analyses together with basic properties of the loops as machine learning \textit{features}.
%
The loop parallelizability model is trained, validated, and tested on 1415 loops
from the SNU NAS Parallel Benchmarks (SNU
NPB)~\cite{Seo:2011:PCN:2357490.2358063}.
%
The loops are labelled using a combination of expert OpenMP~\cite{Dagum:1998:OIA:615255.615542}
annotations and optimization reports from the Intel \cpp{} Compiler (ICC), a
production-quality parallelizing compiler.
%
The model is evaluated on multiple machine learning algorithms.
%
The evaluation shows that -- despite the limited size of the data set -- our model achieves a prediction accuracy higher than 90\%.
%
\quad The parallelization assistant combines inference on the parallelizability model
with traditional profiling to rank higher those loops with a high probability of being parallelizable and impacting the program performance.
%
An evaluation on eight programs from the SNU NPB suite shows that
the program performance tends to improve faster as loops are parallelized in the ranking order suggested by our parallelization assistant compared to a traditional order based on profiling only.
%
On average, following the order suggested by the assistant reduces by approximately 20\% the number of lines of code a programmer has to examine manually to parallelize SNU NPB to its expert-level speedup.
%
Given the high level of effort involved in manual analysis, such a reduction
can translate into substantial development cost savings.

\subsection{Contributions}
\quad In summary, the project of our machine learning based loop parallelization assistant makes the following contributions:
%
\begin{itemize}
\renewcommand\labelitemi{$\vartriangleright$}
\renewcommand\labelitemii{$\bullet$}
\item We introduce a machine learning model, which can be used to predict the probability with which sequential C loops can be parallelized (Sections~\ref{predicting_parallel_loops} and~\ref{ml_predictive_performance});
\item we integrate profiling of execution time with our novel ML model into a parallelization assistant, which guides the user through a ranked list of loops for parallelization (Section~\ref{practical_applications}); and
\item we demonstrate that our tool and methodology increase programmer productivity by identifying parallel loop candidates better than existing state-of-the-art approaches (Section~\ref{evaluation}).
\end{itemize}


\section{Computational frameworks library}
\label{introduction_frameworks}

% The problem

%% Data-centric parallelization problem

\quad Among all lower level implementation questions, the problem of successfull data structure choice stands particularly prominent and important. Listings \ref{lst:introduction_array} and \ref{lst:introduction_list} illustrate the how easily the parallelization can be hampered.
\begin{minipage}[t]{0.4\linewidth}
\begin{lstlisting}[caption={Parallelisable loop operating on a \textbf{linear array}.},label={lst:introduction_array},language=C]
for (int i=0; i<n; it++) {
  a[i]=a[i]+1;
}
\end{lstlisting}
\end{minipage}
%
\begin{minipage}[t]{0.6\linewidth}
\begin{lstlisting}[caption={Non-parallelisable loop operating on a \textbf{linked-list}.},label={lst:introduction_list},language=C]
for (p=list; p!=NULL; p=p->next) {
  p->value+=1;
}
\end{lstlisting}
\end{minipage}


%% Literature review: unavailability of automatic data structure recognizers

addressing the problem of unsuccessful data structure choice,
 The tool is at the literature review and feasibility study stage.
 
%% SPEC CPU2006 complexity level example

% The solution
There are real world benchmarks demonstrating that it is not always really necessary to separate data structures from the algorithms.

The library implements the novel notion of \textit{computational frameworks}. The latter help a programmer with a coarse-grained parallelization tasks, such as software design, algorithm and data structure choice. The library provides a user with a well-designed, modern and convenient interface. We prototyped the library on the suite of Olden benchmarks. The parallel library version is consistently outperforming the sequential version hitting 5-6x speedups on the major benchmarks.\newline\null

\quad The programmer starts on the high level of problem domain and algorithm. The task is to find independent parts in the problem to be solved in parallel. If there are dependencies among them, then the programmer has to design a synchronization mechanism. In our solution package we propose an off-the-shelf C++ template library of computational frameworks. The latter are ready to use parallelism patterns. A programmer only needs to customize his computation through an LLVM-like interface.

\subsection{Contributions}
\begin{itemize}
\renewcommand\labelitemi{$\vartriangleright$}
\renewcommand\labelitemii{$\bullet$}
\item We introduce a machine learning model, which can be used to predict the probability with which sequential C loops can be parallelized (Sections~\ref{predicting_parallel_loops} and~\ref{ml_predictive_performance});
\item we integrate profiling of execution time with our novel ML model into a parallelization assistant, which guides the user through a ranked list of loops for parallelization (Section~\ref{practical_applications}); and
\item we demonstrate that our tool and methodology increase programmer productivity by identifying parallel loop candidates better than existing state-of-the-art approaches (Section~\ref{evaluation}).
\end{itemize}

